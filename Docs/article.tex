\documentclass[11pt]{article}
\usepackage{amsmath,amssymb,amsthm}
\usepackage{hyperref}
\usepackage{geometry}
\geometry{margin=1in}
\title{The Qadmon Universal Criticality Theory (QUCT):\\
A Unified Analytical Law for Spectral Criticality and the Origin of Random-Matrix Statistics}
\author{Igor Tchetchelnitsky\\Independent Researcher, Ashkelon, Israel\\\texttt{igor.tchetch@protonmail.com}\\ORCID: 0009-0007-4607-1946}
\date{20 November 2025}

\begin{document}
\maketitle
\begin{abstract}
We present the Qadmon Universal Criticality Theory (QUCT), a self-contained mathematical framework that defines a universal criticality parameter \ensuremath{\gamma^*} governing the transition of spectra into Random Matrix Theory (RMT) statistics. QUCT models spectral coherence by an integral variational functional F(\ensuremath{\gamma}) and identifies the critical point as the unique root of F^{(3)}(\ensuremath{\gamma})=0. For Dyson index \ensuremath{\beta}=2 (GUE) the theory predicts \ensuremath{\gamma^*}\ensuremath{\approx} 0.3958242245; numerical verification on \ensuremath{10^6} nontrivial Riemann zeros confirms this value to high precision. We include full derivations, proofs of key lemmas, numerical protocols, and reproducible code.
\end{abstract}
\tableofcontents
\section{Introduction}
\section{Mathematical Framework: Integral QUCT Functional}
\section{Analytical derivation of \ensuremath{\gamma^*}(\ensuremath{\beta})}
\section{Numerical verification on Riemann zeros}
\section{Discussion and conclusions}
\bibliographystyle{plain}
\end{document}
